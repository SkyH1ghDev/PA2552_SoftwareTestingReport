\documentclass{article}

\bibliographystyle{IEEEtran}

\title{PA2552 - Software Testing \\
	\large Lean Testing Principles
	}

\author{Christoffer Bohman \\
	MScEng: Game Engineering \\
	Blekinge Institute of Technology
	}

\date{\today}



\begin{document}

\maketitle

\newpage

\section{Introduction}
Testing is an essential part in most large-scale projects, whether that would be in industrial manufacturing plants or 
software development. The practice of ensuring functionality, reliability and safety of crucial mechanisms is a foundational 
requirement for any kind of development. There are multiple ways of evaluating different types of metrics for a variety of 
use cases, some wider known types include: stress testing, performance testing and smoke testing. \cite{IBMSoftwareTesting}

The purpose of this technical report is to collect, analyse and summarise relevant information touching on the
subject of lean software testing. More specifically, this report aims to answer the following questions:

\begin{itemize}
	\item What do I, the author, think are the most important principles of lean software testing and why they should be considered important
	\item In which situations can the principles of lean software testing be applicable in general?
	\item In which situations can the principles of lean software testing be applicable and how would they be applied for me, the author?
\end{itemize}

\section{Methodology}

\section{Results}

\section{Analysis}

\section{Discussion}

\section{Conclusions}



\bibliography{refs}

\end{document}