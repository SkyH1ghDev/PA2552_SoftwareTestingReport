\documentclass{article}
\usepackage{hyperref}

\bibliographystyle{IEEEtran}

\title{PA2552 - Software Testing \\
	\large Lean Testing Principles
	}

\author{Christoffer Bohman \\
	MScEng: Game Engineering \\
	Blekinge Institute of Technology
	}

\date{\today}

\begin{document}

\maketitle

\paragraph{
	\begin{center}
		Notice of AI usage:
	\end{center}
	\begin{center}
		Scopus "AI Query Builder" provided by Blekinge Institute of Technology has been used for the purpose
		of building database queries in search of relevant articles. 
	\end{center}
}
\newpage

\section{Introduction} \label{Introduction}
Testing is an essential part in most large-scale projects, whether that would be in industrial manufacturing plants or 
software development. The practice of ensuring functionality, reliability and safety of crucial mechanisms is a foundational 
requirement for any kind of development. There are multiple ways of evaluating different types of metrics for a variety of 
use cases, some wider known types include: stress testing, performance testing and smoke testing \cite{IBMSoftwareTesting}.

The purpose of this technical report is to collect, analyse and summarise relevant information touching on the
subject of lean software testing. More specifically, this report aims to answer the following questions:

\begin{itemize}
	\item What do I, the author, think are the most important principles of lean software testing and why they should be considered important
	\item In which situations can the principles of lean software testing be applicable in general?
	\item In which situations can the principles of lean software testing be applicable and how would they be applied for me, the author?
\end{itemize}

\section{Methodology}

The research method employed by this report is to search the common internet for information that may be 
relevant for the questions outlined in the introduction (section \ref{Introduction}). Various data gathering
methods have been utilised, namely the \href{https://www.google.com}{Google Search Engine} and the \href{https://www.scopus.com}{Scopus Journal Database} have both been
used for said gathering. The vetting of the search results was mainly comprised of filtering the contents based 
on keywords like "agile testing", "lean testing", "software development", "game development" and "software engineering".

\section{Results}

Lean testing is based on principles that closely align to the commonly known principles and 
values of agile and lean where the purpose is to maximise efficiency, adapt to change and minimise
waste. The goals of lean testing according to Jain \cite{testsigmaLeanTesting}, can be reduced to:

\begin{enumerate}
	\item Value Delivery
	\item Waste Reduction
	\item Efficiency
	\item Continuous Improvement
	\item Customer Satisfaction
\end{enumerate}

\section{Analysis}

\section{Discussion}

\section{Conclusions}

\newpage

\bibliography{refs}

\end{document}