\documentclass{article}
\usepackage{hyperref}
\usepackage{array}
\usepackage{booktabs, caption}
\usepackage[flushleft]{threeparttable}
\usepackage{float}
\usepackage[top=2.5cm, bottom=2.5cm, left=2.5cm, right=2.5cm]{geometry}

\bibliographystyle{IEEEtran}

\title{PA2552 - Software Testing \\
	\large Lean Testing Principles
	}

\author{Christoffer Bohman \\
	MScEng: Game Engineering \\
	Blekinge Institute of Technology \\
	\\
	Contact: \\
	chbh22@student.bth.se / student@skyh1gh.dev \\
	\\
	Revisions and git history: \\
	\href{https://github.com/SkyH1ghDev/PA2552_SoftwareTestingReport}{GitHub} \\
	}

\date{\today}

\begin{document}

\maketitle


\newpage

\paragraph{
	\begin{center}
		Notice of AI usage:
	\end{center}
	\begin{center}
		Scopus "AI Query Builder" provided by Blekinge Institute of Technology has been used for the purpose
		of building database queries in search of relevant articles. These queries were manually checked and 
		verified by the author.
	\end{center}
}

\section{Introduction} \label{Introduction}
Testing is an essential part in most large-scale projects, whether that would be in industrial manufacturing plants or 
software development. The practice of ensuring functionality, reliability and safety of crucial mechanisms is a foundational 
requirement for any kind of development. There are multiple ways of evaluating different types of metrics for a variety of 
use cases, some wider known types include: stress testing, performance testing and smoke testing \cite{IBMSoftwareTesting}.

The purpose of this technical report is to collect, analyse and summarise relevant information touching on the
subject of lean software testing. More specifically, this report aims to questions: ``What lean testing principles are there?''
and ``How do they relate to testing?''

\section{Methodology}

The research method employed by this report is to search the common internet for information that may be 
relevant for the questions outlined in the introduction (section \ref{Introduction}). Various data gathering
methods have been utilised, namely the \href{https://www.google.com}{Google Search Engine} and the \href{https://www.scopus.com}{Scopus Journal Database} have both been
used for said gathering. The vetting of the search results was mainly comprised of filtering the contents based 
on keywords like "agile testing", "lean testing", "software development", "game development" and "software engineering".

\section{Results}

Lean testing is based on principles that closely align to the commonly known principles and 
values of agile and lean where the purpose is to maximise efficiency, adapt to change and minimise
waste \cite{testsigmaLeanTesting}. In addition to this, a core perspective that is a result of the lean way 
of thinking is the idea that the value of the product along with customer satisfaction should be
the central focus \cite{testsigmaLeanTesting}

\begin{table}[H]
	\caption{Abstract and Concrete Lean Testing Principles}
	\begin{center}
		\begin{tabular}{ | m{16em} | m{12em} |} 
			\hline
			Abstract & Concrete \\
			\hline
			\hline
			Provide continuous feedback & Efficiency / Effectiveness \\
			Deliver value to the customer & Minimisation \\
			Enable face-to-face communication & Test data generation \\
			Have courage & Execution \\
			Keep it simple & Maintenance \\
			Practice continuous improvement & Values (-illities) \\
			Respond to change & Purpose \\
			Self-organize & Automation \\
			Focus on people & \\
			Enjoy & \\
			\hline
		\end{tabular}
	\end{center}
	\begin{tablenotes}
		\small
		\item Table 1: Tabularised form of the reinterpreted lean testing principles (not lean principles) outlined in the PowerPoint slides 6 \& 9 written by Alégroth and Nass. \cite{PowerPoint}
	\end{tablenotes}
\end{table}

A three-way case-study between waterfall, agile and agile with dynamic QC has been carried out that compared the efficacy
of testing within these methodologies \cite{DynamicQC}. The method Tommy et al. \cite{DynamicQC} employed in their research yielded a result
where the method of doing tests at the end of a production cycle (waterfall) generated a detected defect count of slightly above
twenty. In contrast, agile and agile with dynamic QC reached counts of slightly below eighty and around a hundred and ten respectively.
In addition to this the test case count is around one hundred for waterfall, one thousand two hundred for agile and nine hundred
for agile with dynamic QC \cite{DynamicQC}.

\section{Analysis}

The results of my data collection show that lean testing adds an additional layer to the development phase. Instead of pushing the 
testing to the very end causing a large backlog of tests, they are done continuously with the value of the product and the customer in mind.
The data also shows that there is a gap in efficacy between waterfall and different implementations of agile and lean testing, showing it is
more beneficial to test continously. Dynamic QC as defined in \cite{DynamicQC} furthers this effect by adding testing as a main focus of a project.

\section{Discussion}

At a broader glance, this report not only shows that lean testing can be boiled down to abstract and concrete principles, but also that the lean testing
methodology can have a much greater impact when combating functional defects in projects. What are the implications of these findings? It can can be argued that
the importance of these principles provide a general and usable framework that speeds up the process by for example extracting the purpose of the testing. This can
be further linked to the ideas of practicing continuous improvement through test assessment and usability. Additionally, by keeping it simple, a layer of 
assessment criteria can be derived that try to weed out unnecessarily large and complex tests that don't perform as good as they potentially could.

These patterns can be recognized in, for instance, unit testing. A unit test, as commonly known, should be small and simple in order to test a smaller unit of a large
codebase. Keeping it simple is fairly self-explanatory as unit tests should encompass simpler units and problems. However, depending on the team that has been organised,
the approach to the writing of the tests may be varying. In our small game test, our approach to unit testing was the usage of the Google Test framework where each person 
each had the responsibility of covering their own code. This way of testing required us not only to enable face-to-face communication, but also forced us to have courage 
and put the breaks on the project if the tests didn't satisfy our definition of done. The result of this was having to respond to change as sometimes test cases broke,
and certain discussions had to be held on how certain mechanics should work and why.

\section{Conclusions}

In summary, lean testing is a testing equivalent of the lean methodology with principles that are based around customer satisfaction and value generation.
Lean testing does not necessarily imply a certain type of tests, but rather the way of working with tests. Some argue that this way of interacting with testing
not only increases defect detection, but may also lead to fewer, more precise tests to be maintained.

Some considerations to take into account may be the increased workload of continously executing, adding, changing and updating tests to a codebase which in turn
could also impede progress in a few situations. The general idea though, is that with the right design, architecture and a mindset to reduce waste, these disadvantages
will diminish considerably.

\newpage

\bibliography{refs}

\end{document}