\documentclass{article}
\usepackage{hyperref}
\usepackage{array}
\usepackage{booktabs, caption}
\usepackage[flushleft]{threeparttable}
\usepackage{float}
\usepackage[top=2.5cm, bottom=2.5cm, left=2.5cm, right=2.5cm]{geometry}

\bibliographystyle{IEEEtran}

\title{PA2552 - Software Testing \\
	\large Lean Testing Principles
	}

\author{Christoffer Bohman \\
	MScEng: Game Engineering \\
	Blekinge Institute of Technology
	}

\date{\today}

\begin{document}

\maketitle

\paragraph{
	\begin{center}
		Notice of AI usage:
	\end{center}
	\begin{center}
		Scopus "AI Query Builder" provided by Blekinge Institute of Technology has been used for the purpose
		of building database queries in search of relevant articles. 
	\end{center}
}
\newpage

\section{Introduction} \label{Introduction}
Testing is an essential part in most large-scale projects, whether that would be in industrial manufacturing plants or 
software development. The practice of ensuring functionality, reliability and safety of crucial mechanisms is a foundational 
requirement for any kind of development. There are multiple ways of evaluating different types of metrics for a variety of 
use cases, some wider known types include: stress testing, performance testing and smoke testing \cite{IBMSoftwareTesting}.

The purpose of this technical report is to collect, analyse and summarise relevant information touching on the
subject of lean software testing. More specifically, this report aims to questions: ``What are the most important principles of 
lean'' and ``where can these principles be applied?''

\section{Methodology}

The research method employed by this report is to search the common internet for information that may be 
relevant for the questions outlined in the introduction (section \ref{Introduction}). Various data gathering
methods have been utilised, namely the \href{https://www.google.com}{Google Search Engine} and the \href{https://www.scopus.com}{Scopus Journal Database} have both been
used for said gathering. The vetting of the search results was mainly comprised of filtering the contents based 
on keywords like "agile testing", "lean testing", "software development", "game development" and "software engineering".

\section{Results}

Lean testing is based on principles that closely align to the commonly known principles and 
values of agile and lean where the purpose is to maximise efficiency, adapt to change and minimise
waste \cite{testsigmaLeanTesting}. In addition to this, a core perspective that is a result of the lean way 
of thinking is the idea that the value of the product along with customer satisfaction should be
the central focus \cite{testsigmaLeanTesting}

\begin{table}[H]
	\caption{Lean Testing Principles}
	\begin{center}
		\begin{tabular}{ | m{12em} |} 
			\hline
			Efficiency / Effectiveness \\
			Minimisation \\
			Test data generation \\
			Execution \\
			Maintenance \\
			Values (-illities) \\
			Purpose \\
			Automation \\
			\hline
		\end{tabular}
	\end{center}
	\begin{tablenotes}
		\small
		\item Table 1: Tabularised form of the reinterpreted lean testing principles (not lean principles) outlined in the PowerPoint slide 9 written by Alégroth et al. \cite{PowerPoint}
	\end{tablenotes}
\end{table}

A three-way case-study between waterfall, agile and agile with dynamic QC has been carried out that compared the efficacy
of testing within these methodologies \cite{DynamicQC}. The method Tommy et al. \cite{DynamicQC} employed in their research yielded a result
where the method of doing tests at the end of a production cycle (waterfall) generated a detected defect count of slightly above
twenty. In contrast, agile and agile with dynamic QC reached counts of slightly below eighty and around a hundred and ten respectively.
In addition to this the test case count is around one hundred for waterfall, one thousand two hundred for agile and nine hundred
for agile with dynamic QC \cite{DynamicQC}.

\section{Analysis}

The results of my data collection show that lean testing adds an additional layer to the development phase. Instead of pushing the 
testing to the very end causing a large backlog of tests, they are done continuously with the value of the product and the customer in mind.
The data also shows that there is a gap in efficacy between waterfall and different implementations of agile and lean testing, showing it is
more beneficial to test continously. Dynamic QC as defined in \cite{DynamicQC} furthers this effect by adding testing as a main focus of a project.

\section{Discussion}

How could this impact software engineering, and more importantly, how could this affect me and my continued studies and projects? 
Lean testing is most applicable in larger projects, often showing minor improvements for smaller projects. However as previously stated, lean testing
has shown to improve value-creation and customer satisfaction as a result of increase defect detections in addition to being adaptable.

My perspective on the most important lean testing principle would be the principle of purpose with the main reason being that without purpose, 
there is no need for the test. Applying this principle on a higher level also helps the efficiency, execution and automation aspects by allowing
for the categorisation of tests. This in turn allows for test execution only for the relevant parts of a code base, further allowing independant 
development between individuals and groups.

A scenario where these principles would be especially applicable for me would be in the upcoming Large Game Project where the development process 
will be of an agile kind. However, since our use case would be rather specific, a variety of tests could be of use. For example, engine-specific
testing could adhere to what most would assume to be standard testing, however rendering and GUI would have to be handeled in another way.

\section{Conclusions}

In summary, lean testing is a testing equivalent of the lean methodology with principles that are based around customer satisfaction and value generation.
Lean testing does not necessarily imply a certain type of tests, but rather the way of working with tests. Some argue that this way of interacting with testing
not only increases defect detection, but may also lead to fewer, more precise tests to be maintained.

Some considerations to take into account may be the increased workload of continously executing, adding, changing and updating tests to a codebase which in turn
could also impede progress in a few situations. The general idea though, is that with the right design, architecture and a mindset to reduce waste, these disadvantages
will diminish considerably.

\newpage

\bibliography{refs}

\end{document}